\documentclass[letterpaper,10pt]{article}
\usepackage{fancyhdr}
\usepackage{amsmath}
\usepackage{amssymb}
\usepackage{bm}
\usepackage{graphicx}
\usepackage{enumerate}
\usepackage{caption}
% \usepackage{natbib}
\usepackage{subcaption}
\usepackage{hyperref}
\usepackage{enumerate}
\usepackage[margin=1in]{geometry}
\pagestyle{fancy}
\lhead{Jesse Mu, Andrew Francl}
\rhead{CSCI3341 Project Proposal}

\begin{document}

\section*{Imitating famous artists with convolutional neural networks}

We propose to develop a machine learning system modeled after an algorithm in a
recent paper by Gatys et al \cite{gatys15}. The algorithm in the paper
describes a convolutional neural network algorithm that enables any given
photograph to be recreated in the style of famous paintings. Specifically, the
algorithm is able to take a painting and learn its stylistic composition rather
than its content, then apply this stylistic information to a new image to
create a new version of the image that stylistically matches the original
painting. We have multiple hopes for this project which depend on how much we
are able to complete in our specified time:

\begin{itemize}
  \item (Fallback) Create a command-line tool using the Caffe \cite{jia14} deep learning
    framework to recreate the algorithm demonstrated in the paper. Open source
    implementations of the algorithm using Lua and Torch do
    exist\footnote{e.g.\ \href{https://github.com/jcjohnson/neural-style}{https://github.com/jcjohnson/neural-style}}
    but we'll be building an original implementation starting from the paper
    in Python to gain a better understanding of how the system works.
  \item (Target) Expand upon the original algorithm in the paper by attempting
    to recreate images not in the style of a single famous painting but by the
    style of an artist oeuvre. This will require either 1) learning a stylistic
    representation of an artist's lifetime work by training a network on not
    just one, but many paintings, or 2) stochastically chosing a painting or
    set of paintings on which to recreate a picture.
  \item (Stretch) Create a web application that allows for any user to choose
    an artist, upload an image, and obtain a stylistically transformed version
    of the image. The limiting factor I forsee is that the computation required
    for an individual picture may be too much to make this website available to
    the public without paying for a cloud computing service like Amazon EC2.
\end{itemize}

Deliverables include a GitHub repository of all of the work we'll complete this
semester, including a pdf writeup of the steps taken in implementing our system, the Python and Caffe code, and instructions and examples.

\subsection*{Helpful topics from the AI course}

\begin{itemize}
    \item Machine learning, and specifically artificial neural networks. We will be
    building particularly advanced models of ANNs to handle image processing tasks.
    \item Probability can be used to incorporate random elements into the
      processing of the images (e.g.\ controlling style versus content
      tradeoffs, perhaps randomly selecting paintings)
\end{itemize}

\subsection*{Challenges}

\begin{itemize}
    \item The principal challenge is not only understanding the paper's
      algorithm, but extending it to incorporate the stylistic content of a
      complete body of images. From a technical perspective, this involves
      training the CNN in charge of handling style with many images of a single
      artist, which will also require a fundamental understanding of how the
      network works.
    \item Conceptually, it's unclear whether the stylistic information derived
      from training the network on several images will result in a general
      sense of an artist's style, or a confusing and meaningless average of an
      artist's work. For example, some artists vary in style considerably over
      their careers. If the latter, we need to make decisions about
      how to obtain a sensible stylistic representation. It may be useful to
      limit the training images to a specific subset of similar paintings.
    \item One challenge will be optimizing computation time. CNNs are
      computationally intensive and can take significant time to train.
      Optimizations on both the software level, such as saving model
      parameters, and the hardware level, such as enabling CUDA GPU
      computation, must be taken into account when designing our system.
\end{itemize}


\nocite{*} % Include all references in bib
\bibliographystyle{annotate}
\bibliography{proposal}{}

\end{document}
